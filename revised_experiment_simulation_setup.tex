\section*{EXPERIMENT / SIMULATION SETUP}

\subsection*{Project Overview and System Architecture Vision}

The multi-agent formation control system represents a comprehensive research and development platform designed for advanced autonomous drone coordination. The project architecture encompasses both current operational capabilities and strategic development plans for future system enhancement. The overall system design follows a modular, scalable approach that facilitates both immediate research validation and long-term operational deployment.

The simulation environment is constructed using a hybrid approach combining standalone C++ simulation capabilities with planned integration into ROS 2 (Humble) middleware infrastructure. This design strategy enables immediate research progress through the robust standalone implementation while maintaining clear migration paths toward distributed, node-based ROS 2 architecture for enhanced modularity and real-world deployment capabilities.

The current implementation centers around the \texttt{agent\_control\_pkg}, which serves as the primary development and testing platform. This package contains fully operational implementations of advanced control algorithms including Proportional-Integral-Derivative (PID) controllers, General Type-2 Fuzzy Logic Systems (GT2-FLS), and comprehensive system integration frameworks. The standalone C++ architecture provides deterministic, high-performance execution essential for control algorithm validation and parameter optimization.

Future system architecture plans include full integration with the Webots 3D robotic simulator, selected for its robust physics engine, accurate aerial robotics modeling capabilities, and native ROS 2 support. Webots will facilitate realistic simulations incorporating environmental disturbances such as wind gusts, sensor noise, and inter-agent communication constraints, thereby enabling comprehensive testing and validation of control strategies under realistic operational conditions without requiring physical prototypes.

\subsection*{Current Implementation Architecture and Status}

The operational simulation environment is implemented as a sophisticated standalone C++ system within the \texttt{agent\_control\_pkg}, providing a comprehensive testing platform for advanced control strategies. The system architecture centers around the main executable \texttt{multi\_drone\_pid\_test\_main.cpp}, which orchestrates complex simulations of three quadcopter drones executing coordinated formation control maneuvers under various environmental and operational conditions.

The current system represents a fully functional, production-ready implementation capable of executing comprehensive multi-scenario testing campaigns. This implementation provides immediate research validation capabilities while serving as the foundational codebase for future ROS 2 integration efforts. The modular design ensures that core control algorithms and system components can be seamlessly migrated to distributed architectures as the project evolves.

\subsubsection*{System Component Structure}

The current implementation encompasses several critical subsystems:

\begin{itemize}
    \item \textbf{Control Algorithm Core}: Advanced PID and GT2-FLS implementations with sophisticated anti-windup, derivative filtering, and feed-forward capabilities
    \item \textbf{Drone Dynamics Simulation}: Realistic quadcopter physics modeling including mass, drag, motor response delays, and acceleration constraints
    \item \textbf{Environmental Modeling}: Comprehensive wind disturbance simulation with constant, sinusoidal, and multi-axis perturbations
    \item \textbf{Formation Coordination}: Multi-phase mission execution with dynamic formation reconfiguration and waypoint navigation
    \item \textbf{Performance Assessment}: Real-time metrics computation including RMS errors, settling times, overshoot analysis, and stability evaluation
    \item \textbf{Data Logging and Analysis}: Comprehensive state logging with automated post-processing and visualization capabilities
\end{itemize}

\subsection*{Mathematical Framework and Control Theory Implementation}

\subsubsection*{PID Controller Mathematical Foundation}

The core PID controller implements a sophisticated discrete-time control law incorporating advanced features for robust performance under challenging operational conditions. The fundamental control equation is:

\begin{equation}
u(k) = K_p e(k) + K_i \sum_{j=0}^{k} e(j) \Delta t + K_d \frac{m(k) - m(k-1)}{\Delta t}
\end{equation}

where $e(k) = r(k) - m(k)$ represents the tracking error, $r(k)$ is the reference setpoint, $m(k)$ is the measured process variable, and the derivative term is computed on the measurement to eliminate derivative kick phenomena during setpoint changes.

The implementation incorporates several advanced control features essential for high-performance formation control:

\begin{itemize}
    \item \textbf{Enhanced Anti-windup Protection}: Conditional integration with saturation detection prevents integral windup during output limitation:
    \begin{equation}
    I(k+1) = \begin{cases}
    I(k) + e(k) \Delta t & \text{if } |u(k)| < u_{\max} \text{ and not saturated} \\
    I(k) - \alpha \cdot \text{excess\_output} / K_i & \text{if output clamped}
    \end{cases}
    \end{equation}
    where $\alpha = 0.5$ provides partial back-calculation to prevent oscillation.

    \item \textbf{Derivative Noise Filtering}: A first-order low-pass filter significantly reduces measurement noise impact:
    \begin{equation}
    D_f(k) = \alpha D_{raw}(k) + (1-\alpha) D_f(k-1)
    \end{equation}
    where $\alpha = 0.2$ represents the filter coefficient optimized for control bandwidth preservation while achieving effective noise attenuation.

    \item \textbf{Feed-forward Control Enhancement}: Predictive control improves tracking performance during rapid setpoint changes:
    \begin{equation}
    u_{total}(k) = u_{PID}(k) + K_{ff,v} \dot{r}(k) + K_{ff,a} \ddot{r}(k)
    \end{equation}
    with velocity feed-forward gain $K_{ff,v} = 0.8$ and acceleration feed-forward gain $K_{ff,a} = 0.1$.
\end{itemize}

\subsubsection*{General Type-2 Fuzzy Logic System Mathematical Framework}

The GT2-FLS implementation follows a rigorous mathematical framework for interval Type-2 fuzzy inference, providing enhanced uncertainty handling capabilities compared to conventional Type-1 fuzzy systems. The complete inference process consists of four distinct computational stages:

\begin{enumerate}
    \item \textbf{Fuzzification Stage}: Crisp input variables $(e, \dot{e}, w)$ representing tracking error, error derivative, and wind disturbance are mapped to interval Type-2 membership functions characterized by upper and lower membership bounds:
    \begin{equation}
    \tilde{A}_i(x) = [\underline{\mu}_{\tilde{A}_i}(x), \overline{\mu}_{\tilde{A}_i}(x)]
    \end{equation}

    Each membership function is defined by six parameters $[l_1, l_2, l_3, u_1, u_2, u_3]$ representing triangular lower and upper membership function bounds, enabling precise uncertainty representation in the fuzzy inference process.

    \item \textbf{Rule Evaluation and Firing Strength Computation}: For each fuzzy rule $R^l$ of the form "IF $x_1$ is $\tilde{A}_1^l$ AND $x_2$ is $\tilde{A}_2^l$ AND $x_3$ is $\tilde{A}_3^l$ THEN $y$ is $\tilde{C}^l$", the firing strength interval is computed using the minimum t-norm operation:
    \begin{equation}
    \tilde{f}^l = [\underline{f}^l, \overline{f}^l] = [\min(\underline{\mu}_1^l, \underline{\mu}_2^l, \underline{\mu}_3^l), \min(\overline{\mu}_1^l, \overline{\mu}_2^l, \overline{\mu}_3^l)]
    \end{equation}

    \item \textbf{Type Reduction via Karnik-Mendel Algorithm}: The type-reduced set computation aggregates all fired rule contributions using the iterative Karnik-Mendel procedure:
    \begin{equation}
    Y_{tr} = [y_l, y_r] = \left[\frac{\sum_{i=1}^{L} f_i^l y_i^l}{\sum_{i=1}^{L} f_i^l}, \frac{\sum_{i=1}^{R} f_i^r y_i^r}{\sum_{i=1}^{R} f_i^r}\right]
    \end{equation}

    where the left and right end-points are determined through iterative switching point identification ensuring optimal type reduction accuracy.

    \item \textbf{Defuzzification}: The final crisp output is obtained through centroid computation:
    \begin{equation}
    y = \frac{y_l + y_r}{2}
    \end{equation}
\end{enumerate}

\subsection*{Comprehensive Fuzzy Logic Rule Base and Linguistic Framework}

The GT2-FLS employs a carefully designed rule base consisting of 21 expert-derived rules covering comprehensive operational scenarios. The linguistic variable structure provides intuitive yet mathematically precise representation of control knowledge:

\begin{itemize}
    \item \textbf{Position Error}: NB (Negative Big), NS (Negative Small), ZE (Zero), PS (Positive Small), PB (Positive Big)
    \item \textbf{Error Derivative}: DN (Derivative Negative), DZ (Derivative Zero), DP (Derivative Positive)
    \item \textbf{Wind Disturbance}: SNW (Strong Negative Wind), WNW (Weak Negative Wind), NWN (No Wind/Neutral), WPW (Weak Positive Wind), SPW (Strong Positive Wind)
    \item \textbf{Control Output}: XLNC (Extra Large Negative Correction), LNC (Large Negative Correction), SNC (Small Negative Correction), NC (No Correction), SPC (Small Positive Correction), LPC (Large Positive Correction), XLPC (Extra Large Positive Correction)
\end{itemize}

Representative rules demonstrate the system's intelligent compensation strategies:
\begin{itemize}
    \item IF Error is PB AND dError is DZ AND Wind is NWN THEN Correction is LPC (Large positive correction for significant positive error)
    \item IF Error is ZE AND dError is DZ AND Wind is SPW THEN Correction is LNC (Compensate for positive wind when at setpoint)
    \item IF Error is ZE AND dError is DZ AND Wind is SNW THEN Correction is LPC (Compensate for negative wind when at setpoint)
    \item IF Error is NS AND dError is DP AND Wind is NWN THEN Correction is SNC (Small correction when error is decreasing)
\end{itemize}

\subsection*{Realistic Drone Dynamics and Physical Modeling}

Each quadcopter drone is modeled as a six-degree-of-freedom rigid body with realistic physical constraints and dynamic characteristics representative of commercial UAV platforms. The fundamental equations of motion for translational dynamics are:

\begin{align}
\ddot{x} &= \frac{1}{m}(u_x + F_{wind,x} - D_x) \\
\ddot{y} &= \frac{1}{m}(u_y + F_{wind,y} - D_y)
\end{align}

where the system incorporates several critical physical modeling elements:

\begin{itemize}
    \item \textbf{Vehicle Mass}: $m = 1.5$ kg, representative of mid-size quadcopter platforms
    \item \textbf{Control Input Constraints}: $u_{x,y} \in [-15, +15]$ m/s², reflecting realistic motor/propeller acceleration capabilities
    \item \textbf{Aerodynamic Drag}: $D_{x,y} = 0.1 \cdot v^2 \cdot \text{sign}(v)$, modeling quadratic air resistance
    \item \textbf{Motor Response Dynamics}: First-order lag with time constant $\tau = 0.1$ s, simulating ESC and motor response delays
    \item \textbf{Velocity Constraints}: Maximum velocity limited to 20 m/s for operational safety
\end{itemize}

The motor response delay is implemented as:
\begin{equation}
\frac{d u_{commanded}}{dt} = \frac{1}{\tau}(u_{desired} - u_{commanded})
\end{equation}

This comprehensive physical modeling ensures simulation results accurately reflect real-world drone behavior and control system performance.

\subsection*{Advanced Ziegler-Nichols Tuning Implementation}

The system incorporates a sophisticated automated Ziegler-Nichols tuning framework providing multiple operational modes for comprehensive PID parameter optimization. This implementation significantly enhances the traditional ZN methodology through automation and extended analysis capabilities.

\subsubsection*{Operational Modes}

\begin{enumerate}
    \item \textbf{Auto-Search Mode}: Systematically explores the parameter space by varying $K_p$ from 10.0 to 200.0 in 5.0 increments to identify the ultimate gain $K_u$ and oscillation period $P_u$. The algorithm monitors system response characteristics including overshoot percentage, settling time, and oscillation period to determine critical stability boundaries.

    \item \textbf{Manual Testing Mode}: Enables testing of specific $K_p$ values with $K_i = K_d = 0$ to observe pure proportional response characteristics and oscillation behavior. This mode facilitates detailed analysis of individual parameter effects and system stability margins.

    \item \textbf{Quick Method Application}: Implements six different ZN tuning methodologies, each optimized for specific performance requirements:
    \begin{itemize}
        \item \textbf{Classic ZN}: $K_p = 0.6K_u$, $T_i = 0.5P_u$, $T_d = 0.125P_u$ (Balanced performance)
        \item \textbf{Conservative}: $K_p = 0.33K_u$, $T_i = 0.5P_u$, $T_d = 0.33P_u$ (Enhanced stability)
        \item \textbf{Less Overshoot}: $K_p = 0.45K_u$, $T_i = 0.8P_u$, $T_d = 0.12P_u$ (Reduced overshoot)
        \item \textbf{No Overshoot}: $K_p = 0.2K_u$, $T_i = 0.5P_u$, $T_d = 0.33P_u$ (Critical damping)
        \item \textbf{Aggressive}: $K_p = 0.8K_u$, $T_i = 0.4P_u$, $T_d = 0.1P_u$ (Fast response)
        \item \textbf{Modern ZN}: $K_p = 0.5K_u$, $T_i = 0.75P_u$, $T_d = 0.15P_u$ (Contemporary interpretation)
    \end{itemize}
\end{enumerate}

The current optimal parameters ($K_p = 3.1$, $K_i = 0.4$, $K_d = 2.2$) were derived through extensive ZN analysis and represent an excellent compromise between response speed, stability margins, and disturbance rejection capabilities.

\subsection*{Formation Control Coordination and Mission Architecture}

The system implements sophisticated multi-phase formation control scenarios designed to evaluate comprehensive operational capabilities under varying conditions. The formation coordination framework manages triangular formation geometry while executing complex waypoint navigation sequences.

\subsubsection*{Mission Phase Structure}

The standard evaluation mission consists of four distinct operational phases, each designed to test specific control system capabilities:

\begin{enumerate}
    \item \textbf{Phase 1 (0-30s)}: Formation establishment and initial coordination at center point $(5, 5)$ m
    \item \textbf{Phase 2 (30-60s)}: Coordinated transition to waypoint $(-5, 0)$ m, testing formation maintenance during lateral movement
    \item \textbf{Phase 3 (60-90s)}: Navigation to position $(0, -5)$ m, evaluating diagonal formation translation
    \item \textbf{Phase 4 (90-120s)}: Final positioning at $(5, 0)$ m, assessing long-term formation stability
\end{enumerate}

\subsubsection*{Formation Geometry and Initial Conditions}

The triangular formation maintains a 4.0 m side length throughout all mission phases, providing sufficient spacing for safe operation while enabling clear formation identification. Initial drone deployment positions are strategically selected to minimize formation establishment time:

\begin{itemize}
    \item \textbf{Drone 0 (Leader)}: $(0, 1.15)$ m - Forward formation position
    \item \textbf{Drone 1 (Follower)}: $(-2, -2)$ m - Left wing position
    \item \textbf{Drone 2 (Follower)}: $(2, -2)$ m - Right wing position
\end{itemize}

This configuration enables rapid convergence to desired formation geometry while providing balanced initial conditions for comprehensive controller evaluation.

\subsection*{Environmental Disturbance Modeling and Wind Effects}

Environmental disturbances represent critical challenges for formation control systems, requiring robust control strategies capable of maintaining coordination under varying external conditions. The simulation incorporates sophisticated wind modeling combining multiple disturbance types for comprehensive system evaluation.

\subsubsection*{Mathematical Wind Model}

The wind disturbance framework implements both deterministic and stochastic components:
\begin{equation}
F_{wind}(t) = F_{base} + A \sin(\omega t + \phi) + \eta(t)
\end{equation}

where $F_{base}$ represents constant wind forces, $A \sin(\omega t + \phi)$ models periodic gusts, and $\eta(t)$ introduces stochastic turbulence components.

\subsubsection*{Phase-Specific Disturbance Profiles}

Each mission phase incorporates distinct wind patterns designed to evaluate specific control system capabilities:

\begin{itemize}
    \item \textbf{Phase 1 Disturbances}: Constant lateral force (3.0 N) and vertical perturbation (-1.0 N) during formation establishment
    \item \textbf{Phase 2 Disturbances}: Opposing lateral force (-2.0 N) and positive vertical force (1.5 N) during formation translation
    \item \textbf{Phase 3 Disturbances}: Sinusoidal lateral perturbation (2.0 N amplitude, 1.57 rad/s) combined with step vertical forces
    \item \textbf{Phase 4 Disturbances}: Combined constant and high-frequency disturbances (5.0 N forces, 3.14 rad/s oscillations) for maximum challenge
\end{itemize}

This comprehensive disturbance profile ensures thorough evaluation of control system robustness under realistic operational conditions.

\subsection*{Performance Metrics and Quantitative Analysis Framework}

The system implements a comprehensive performance evaluation framework providing detailed quantitative assessment of control system effectiveness across multiple evaluation criteria. This analysis capability enables objective comparison between different control strategies and parameter configurations.

\subsubsection*{Primary Performance Metrics}

\begin{itemize}
    \item \textbf{Root Mean Square Error}: $\text{RMSE} = \sqrt{\frac{1}{N}\sum_{i=1}^{N} e_i^2}$ - Provides overall tracking accuracy assessment
    \item \textbf{Maximum Absolute Error}: $\text{MAE} = \max(|e_i|)$ - Indicates worst-case performance
    \item \textbf{Settling Time Analysis}: Time required to reach and maintain $\pm 2\%$ and $\pm 5\%$ error bands with enhanced stability requirements
    \item \textbf{Overshoot Percentage}: $\text{OS} = \frac{M_p - M_{ss}}{M_{ss}} \times 100\%$ - Measures transient response characteristics
    \item \textbf{Steady-state Error}: Final tracking accuracy after transient settling
\end{itemize}

\subsubsection*{Current Performance Achievements}

Recent comprehensive testing campaigns have produced the following quantitative results demonstrating system effectiveness:

\begin{itemize}
    \item \textbf{PID Only Configuration}: 1.604 m RMS error baseline performance
    \item \textbf{PID + FLS Configuration}: 1.569 m RMS error (2.2\% improvement over baseline)
    \item \textbf{PID + Wind Disturbance}: 1.650 m RMS error (demonstrates disturbance impact)
    \item \textbf{PID + FLS + Wind}: 1.596 m RMS error (3.3\% improvement under disturbances, demonstrating robust control)
\end{itemize}

These results validate the effectiveness of the GT2-FLS augmentation, particularly under challenging environmental conditions.

\subsection*{Comprehensive Automation Infrastructure and Testing Framework}

The project incorporates an extensive automation infrastructure enabling systematic evaluation across multiple configuration spaces while minimizing manual intervention requirements. This framework significantly enhances research productivity and ensures consistent, reproducible experimental procedures.

\subsubsection*{PowerShell Automation Scripts}

The comprehensive testing framework (\texttt{run\_comprehensive\_simulations\_fixed.ps1}) orchestrates systematic evaluation across 13+ distinct configurations encompassing:

\begin{itemize}
    \item \textbf{Baseline PID Evaluation}: Conservative, optimal, and aggressive parameter sets
    \item \textbf{FLS Effectiveness Studies}: Comparative analysis with and without fuzzy logic augmentation
    \item \textbf{Wind Disturbance Impact Assessment}: Quantification of environmental effect severity
    \item \textbf{Feed-forward Control Validation}: Advanced control feature effectiveness evaluation
    \item \textbf{Full System Integration Testing}: Comprehensive multi-feature validation
    \item \textbf{Ziegler-Nichols Method Comparison}: Systematic evaluation of different tuning approaches
\end{itemize}

\subsubsection*{Data Analysis and Visualization Pipeline}

Post-processing utilizes sophisticated Python-based analysis scripts (\texttt{plot\_simulation\_results.py}) generating comprehensive visualization suites including:

\begin{itemize}
    \item \textbf{Trajectory Analysis}: 2D drone path visualization with formation geometry assessment
    \item \textbf{Position Tracking Charts}: Detailed setpoint following performance for each axis
    \item \textbf{Error Analysis Plots}: Temporal error evolution with RMS computation and statistical analysis
    \item \textbf{Control Signal Decomposition}: PID component breakdown (P, I, D terms) and fuzzy logic corrections
    \item \textbf{Force Analysis}: Comparative visualization of control forces versus environmental disturbances
    \item \textbf{Performance Summary Tables}: Quantitative metric compilation across all tested configurations
\end{itemize}

All simulation data is systematically logged to timestamped CSV files containing complete state information including positions, velocities, errors, control signals, environmental forces, and real-time performance metrics.

\subsection*{System Configuration Management and Flexibility}

The system employs sophisticated configuration management through YAML-based parameter files enabling dynamic adjustment of simulation parameters without source code modifications. This approach enhances experimental flexibility while maintaining configuration traceability.

\subsubsection*{Configuration File Architecture}

\begin{itemize}
    \item \textbf{\texttt{simulation\_params.yaml}}: Master configuration containing PID parameters, simulation timing, formation geometry, wind profiles, and output settings
    \item \textbf{\texttt{fuzzy\_params.yaml}}: Complete GT2-FLS definition including membership functions, linguistic variables, and rule base specifications
    \item \textbf{\texttt{pid\_params.yaml}}: Dedicated PID parameter storage for rapid configuration switching
\end{itemize}

This modular configuration approach facilitates rapid parameter space exploration while maintaining experimental reproducibility and configuration version control.

\subsection*{Future Development Plans and System Evolution}

The current standalone implementation serves as the foundational platform for an ambitious long-term development roadmap encompassing several critical advancement areas designed to enhance system capabilities and operational applicability.

\subsubsection*{ROS 2 Integration Architecture}

The planned ROS 2 migration will establish a distributed, modular system architecture comprising three primary packages:

\begin{itemize}
    \item \textbf{\texttt{agent\_control\_pkg}}: Enhanced individual drone control incorporating current PID and GT2-FLS implementations with ROS 2 node architecture
    \item \textbf{\texttt{formation\_coordinator\_pkg}}: Advanced inter-agent coordination management including communication protocols, formation consensus algorithms, and distributed optimization frameworks
    \item \textbf{\texttt{my\_custom\_interfaces\_pkg}}: Specialized ROS 2 message and service definitions optimized for multi-agent coordination including \texttt{FormationState.msg}, \texttt{UpdateFormation.srv}, and \texttt{DisturbanceAlert.msg}
\end{itemize}

\subsubsection*{Webots Integration and Enhanced Simulation}

Future Webots integration will provide:
\begin{itemize}
    \item \textbf{Realistic Physics Simulation}: Advanced aerodynamics, rotor dynamics, and environmental interaction modeling
    \item \textbf{Sensor Modeling}: GPS noise, IMU drift, communication delays, and packet loss simulation
    \item \textbf{Visual Validation}: 3D formation visualization and behavior verification capabilities
    \item \textbf{Hardware-in-the-Loop Preparation}: Interface development for future physical system integration
\end{itemize}

\subsubsection*{Advanced Control Algorithm Integration}

Planned enhancements include:
\begin{itemize}
    \item \textbf{Particle Swarm Optimization (PSO)}: Automated multi-objective parameter optimization for PID and fuzzy logic systems
    \item \textbf{Model Predictive Control (MPC)}: Advanced predictive control for enhanced formation tracking and constraint handling
    \item \textbf{Adaptive Control Systems}: Real-time parameter adaptation based on system identification and performance monitoring
    \item \textbf{Event-triggered Communication}: Bandwidth-optimized communication strategies for distributed coordination
\end{itemize}

\subsection*{Documentation and Reproducibility Framework}

Comprehensive documentation maintained within the project repository ensures accessibility, reproducibility, and collaborative development capability. The documentation framework includes:

\begin{itemize}
    \item \textbf{Build Instructions}: Detailed compilation procedures for multiple platforms with dependency management
    \item \textbf{Configuration Guides}: Parameter tuning guidelines and configuration best practices
    \item \textbf{API Documentation}: Complete interface specifications for all system components
    \item \textbf{Experimental Procedures}: Standardized testing protocols ensuring consistent result generation
    \item \textbf{Performance Benchmarks}: Reference results enabling system validation and regression testing
\end{itemize}

This comprehensive documentation strategy facilitates knowledge transfer, enables collaborative development, and supports long-term system maintenance and evolution.

\subsection*{Conclusion and System Validation}

The implemented simulation environment represents a sophisticated, production-ready platform for advanced multi-agent formation control research and development. The system successfully combines theoretical rigor with practical implementation considerations, providing comprehensive validation capabilities for control algorithm effectiveness under realistic operational conditions.

The current achievements demonstrate significant advancement in formation control technology, with quantifiable improvements in tracking accuracy, disturbance rejection, and system robustness. The GT2-FLS integration provides measurable performance enhancement while the comprehensive automation infrastructure enables efficient, systematic evaluation across extensive parameter spaces.

This foundational implementation establishes a robust platform for continued research advancement while providing immediate capabilities for comprehensive system validation and performance optimization. The modular design and extensive documentation ensure system sustainability and support future enhancement integration as the multi-agent formation control field continues to evolve.

\subsection*{Detailed Implementation Architecture and File Structure}

The project follows a sophisticated modular architecture designed for maintainability, extensibility, and clear separation of concerns. The file organization reflects industry best practices for large-scale C++ development projects.

\subsubsection*{Core Implementation Files}

\begin{itemize}
    \item \textbf{\texttt{multi\_drone\_pid\_test\_main.cpp}} (1,370 lines): The primary simulation orchestrator implementing the complete testing framework including drone dynamics simulation, control algorithm integration, performance metrics computation, and comprehensive data logging. This file contains the main simulation loop, Ziegler-Nichols tuning algorithms, and automated testing infrastructure.

    \item \textbf{\texttt{pid\_controller.hpp/cpp}} (103/246 lines): Advanced PID controller implementation featuring anti-windup protection, derivative filtering, feed-forward control, and comprehensive diagnostic capabilities. The implementation provides both basic control computation and detailed component breakdown for analysis purposes.

    \item \textbf{\texttt{gt2\_fuzzy\_logic\_system.hpp/cpp}} (85/443 lines): Complete General Type-2 Fuzzy Logic System implementation including interval membership function computation, Karnik-Mendel type reduction, and comprehensive rule inference. The system supports dynamic rule base modification and detailed intermediate result logging.

    \item \textbf{\texttt{config\_reader.hpp}} (135 lines): Sophisticated YAML configuration parsing system enabling dynamic parameter adjustment without recompilation. Supports nested configuration structures, type validation, and default value management.
\end{itemize}

\subsubsection*{Configuration Management System}

The YAML-based configuration system provides hierarchical parameter organization enabling rapid experimental iteration:

\begin{itemize}
    \item \textbf{Simulation Parameters}: Time step (0.1s), total duration (120s), drone count (3), output settings
    \item \textbf{PID Configuration}: Gains ($K_p = 3.1$, $K_i = 0.4$, $K_d = 2.2$), output limits, filtering parameters
    \item \textbf{Formation Geometry}: Side length (4.0m), initial positions, phase transitions, timing
    \item \textbf{Wind Modeling}: Phase-specific force profiles, sinusoidal parameters, disturbance timing
    \item \textbf{ZN Tuning Settings}: Search ranges, convergence criteria, method selection
\end{itemize}

\subsection*{Simulation Workflow and Execution Pipeline}

The simulation execution follows a sophisticated multi-stage pipeline designed for comprehensive testing and analysis:

\subsubsection*{Initialization Phase}
\begin{enumerate}
    \item Configuration file validation and parameter loading
    \item Drone object instantiation with realistic physical parameters
    \item Controller initialization (PID and GT2-FLS systems)
    \item Formation geometry setup and initial positioning
    \item Data logging infrastructure preparation
    \item Performance metrics framework initialization
\end{enumerate}

\subsubsection*{Main Simulation Loop}
The primary execution loop operates at 10 Hz (0.1s time step) implementing:
\begin{enumerate}
    \item Current mission phase determination and setpoint computation
    \item Environmental disturbance application (wind forces)
    \item Control signal computation for each drone (PID + optional FLS)
    \item Drone dynamics integration with realistic physical constraints
    \item Real-time performance metrics computation
    \item Comprehensive state logging to CSV files
    \item Console progress reporting at configurable intervals
\end{enumerate}

\subsubsection*{Post-Processing and Analysis}
Upon simulation completion:
\begin{enumerate}
    \item Final performance metrics calculation and reporting
    \item Comprehensive CSV data export with timestamp identification
    \item Automated Python visualization script execution
    \item Performance summary generation and comparison reporting
\end{enumerate}

\subsection*{Advanced Control Features and Implementation Details}

\subsubsection*{Feed-Forward Control Implementation}

The feed-forward control system anticipates required control effort based on reference trajectory derivatives:
\begin{equation}
u_{ff}(k) = K_{ff,v} \frac{r(k) - r(k-1)}{\Delta t} + K_{ff,a} \frac{r(k) - 2r(k-1) + r(k-2)}{\Delta t^2}
\end{equation}

Current implementation uses $K_{ff,v} = 0.8$ and $K_{ff,a} = 0.1$, providing significant tracking improvement during formation transitions while maintaining stability margins.

\subsubsection*{Enhanced Anti-Windup Strategy}

The anti-windup implementation employs multiple protection mechanisms:
\begin{itemize}
    \item \textbf{Conditional Integration}: Prevents accumulation when output saturated
    \item \textbf{Back-Calculation}: Reduces integral term proportionally to saturation excess
    \item \textbf{Integral Clamping}: Limits maximum integral contribution to 80\% of output range
    \item \textbf{Saturation Detection}: Monitors output limiting for diagnostic purposes
\end{itemize}

\subsection*{Comprehensive Performance Analysis Results}

\subsubsection*{Detailed Performance Comparison}

Recent testing campaigns across multiple scenarios provide quantitative validation of system effectiveness:

\begin{center}
\begin{tabular}{|l|c|c|c|c|}
\hline
\textbf{Configuration} & \textbf{RMS Error (m)} & \textbf{Max Error (m)} & \textbf{Final Error (m)} & \textbf{Improvement} \\
\hline
PID Only & 1.604 & 11.158 & 0.054 & Baseline \\
PID + FLS & 1.569 & 11.155 & 0.052 & +2.2\% \\
PID + Wind & 1.650 & 11.121 & 0.890 & -2.9\% \\
PID + FLS + Wind & 1.596 & 11.179 & 0.522 & +3.3\%* \\
\hline
\end{tabular}
\end{center}
*Improvement relative to PID + Wind configuration

\subsubsection*{GT2-FLS Effectiveness Analysis}

The fuzzy logic system demonstrates particular effectiveness under challenging conditions:
\begin{itemize}
    \item \textbf{Disturbance Rejection}: 3.3\% RMS improvement under wind conditions
    \item \textbf{Transient Response}: Reduced settling time during formation transitions
    \item \textbf{Robustness}: Enhanced stability margins under parameter variations
    \item \textbf{Adaptive Behavior}: Intelligent compensation for environmental uncertainties
\end{itemize}

\subsection*{Scalability and Computational Performance}

The implementation demonstrates excellent computational efficiency enabling real-time operation:

\begin{itemize}
    \item \textbf{Execution Time}: Complete 120s simulation completes in $<$5s wall-clock time
    \item \textbf{Memory Usage}: Minimal heap allocation with stack-based computation
    \item \textbf{Scalability}: Linear complexity scaling with drone count
    \item \textbf{Real-time Capability}: Suitable for hardware-in-the-loop testing
\end{itemize}

\subsection*{Validation and Verification Framework}

The system incorporates comprehensive validation mechanisms ensuring result reliability:

\subsubsection*{Algorithm Validation}
\begin{itemize}
    \item \textbf{Unit Testing}: Individual component validation with known input/output pairs
    \item \textbf{Integration Testing}: Multi-component interaction verification
    \item \textbf{Regression Testing}: Automated validation against reference results
    \item \textbf{Boundary Testing}: Extreme parameter and condition evaluation
\end{itemize}

\subsubsection*{Physical Realism Validation}
\begin{itemize}
    \item \textbf{Energy Conservation}: Verification of dynamic system energy balance
    \item \textbf{Constraint Compliance}: Acceleration and velocity limit enforcement
    \item \textbf{Stability Analysis}: Verification of control system stability margins
    \item \textbf{Disturbance Response}: Realistic environmental interaction modeling
\end{itemize}

This comprehensive validation framework ensures simulation results accurately represent real-world system behavior and provide reliable foundations for control system design decisions.
